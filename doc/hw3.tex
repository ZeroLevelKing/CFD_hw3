\documentclass[UTF8]{ctexart}
\usepackage{amsmath}   
\usepackage{booktabs}  
\usepackage{geometry}  
\usepackage{hyperref}
\usepackage{graphicx} 
\usepackage{float}
\graphicspath{{figure/}} % 指定放置图片的子文件夹路径
\geometry{a4paper, left=2.5cm, right=2.5cm, top=2.5cm, bottom=2.5cm}

\begin{document}

\title{计算流体力学第二次作业}
\author{朱林-2200011028}
\date{\today}
\maketitle

\section{数理算法原理}
对于一阶波动方程,采用三种数值格式进行求解,
分别为Lax-Wendroff格式、Warming-Beam格式和Leap-frog格式。
\subsection{Lax-Wendroff格式}
\subsubsection{离散格式推导}
对一阶波动方程进行泰勒展开至二阶项,时间导数替换为空间导数:
\begin{align}
\frac{\partial u}{\partial t} &= -\frac{\partial u}{\partial x}, \\
\frac{\partial^2 u}{\partial t^2} &= \frac{\partial^2 u}{\partial x^2}.
\end{align}
离散形式为:
\begin{equation}
u_j^{n+1} = u_j^n - \frac{\sigma}{2}(u_{j+1}^n - u_{j-1}^n) + \frac{\sigma^2}{2}(u_{j+1}^n - 2u_j^n + u_{j-1}^n),
\end{equation}
其中 $\sigma = \Delta t / \Delta x$ 为CFL数。

\subsubsection{稳定性分析}
傅里叶模式代入得放大因子:
\begin{equation}
g = 1 - i\sigma \sin(k\Delta x) - \sigma^2(1 - \cos(k\Delta x)).
\end{equation}
稳定性条件为 $|g|^2 \leq 1$,解得 $\sigma \leq 1$。

\subsubsection{精度分析}
截断误差主项为 $O(\Delta t^2, \Delta x^2)$,故为\textbf{二阶精度}。

\subsection{Warming-Beam格式}
\subsubsection{离散格式推导}
迎风三点差分格式:
\begin{equation}
u_j^{n+1} = u_j^n - \sigma(u_j^n - u_{j-1}^n) + \frac{\sigma(\sigma-1)}{2}(u_j^n - 2u_{j-1}^n + u_{j-2}^n).
\end{equation}

\subsubsection{稳定性分析}

将傅里叶模式$u_j^n = g^n e^{ikj\Delta x}$代入格式(5),得到放大因子:

\begin{equation}
g=1-\sigma\left(1-e^{-i k\Delta x}\right)+\frac{\sigma(\sigma-1)}{2}\left(1-2 e^{-i k\Delta x}+e^{-i 2 k\Delta x}\right). 
\end{equation}

计算模平方$|g|^2$,通过极值分析可得稳定性条件为:

$$
0 \leq \sigma \leq 2
$$

当$\sigma=0.5$时,对所有波数$k$均有$|g|^2 \leq 1$,验证格式的稳定性。

\subsubsection{精度分析}
空间差分不对称导致\textbf{二阶精度},伴随显著色散误差。

\subsection{Leap-frog格式}
\subsubsection{离散格式推导}
时间-空间中心差分:
\begin{equation}
u_j^{n+1} = u_j^{n-1} - \sigma(u_{j+1}^n - u_{j-1}^n).
\end{equation}

\subsubsection{稳定性分析}
特征方程解满足 $|g| = 1$,中性稳定条件 $\sigma \leq 1$。

\subsubsection{精度分析}
截断误差主项 $O(\Delta t^2, \Delta x^2)$,\textbf{二阶精度},无耗散但存在相位误差。



\section{代码生成与调试}


\section{结果讨论与物理解释}

\subsection{稳定性验证}
通过数值实验验证三种格式的稳定性特征:

\begin{itemize}
    \item \textbf{Lax-Wendroff 格式}
    \begin{itemize}
        \item $\sigma=0.8$时解保持稳定,波形传播正常
        \item $\sigma=1.1$时出现指数型发散,验证$\sigma\leq1$的必要性
    \end{itemize}
    
    \item \textbf{Warming-Beam 格式}
    \begin{itemize}
        \item $\sigma=1.5$时解保持稳定,但波形出现畸变
        \item $\sigma=2.2$时计算迅速发散,确认$\sigma\leq2$的上限
        \item $\sigma=0.3$时稳定,证明下限$\sigma\geq0$有效
    \end{itemize}
    
    \item \textbf{Leap-frog 格式}
    \begin{itemize}
        \item $\sigma=0.95$时保持中性稳定,但出现寄生振荡
        \item $\sigma=1.05$时振幅持续增长,违反稳定性条件
    \end{itemize}
\end{itemize}

\subsection{精度验证}
通过网格加密实验观察收敛特性:

\begin{itemize}
    \item Lax-Wendroff和Leap-frog格式的误差随$\Delta x$减小呈二次收敛趋势
    \item Warming-Beam格式在$\sigma=1.5$时因稳定性限制收敛速率下降
    \item 三种格式的收敛阶与理论截断误差分析一致
\end{itemize}

\subsection{耗散与相位特性}
对比方波传播的数值解行为:

\begin{itemize}
    \item \textbf{Lax-Wendroff 格式}
    \begin{itemize}
        \item 波形前缘出现轻微过冲,后缘有耗散衰减
        \item 相位滞后现象明显,高频分量传播速度偏慢
    \end{itemize}
    
    \item \textbf{Warming-Beam 格式}
    \begin{itemize}
        \item 强数值耗散导致波峰幅值显著降低
        \item 波形前缘出现超前相位误差
        \item 高频振荡成分被有效抑制
    \end{itemize}
    
    \item \textbf{Leap-frog 格式}
    \begin{itemize}
        \item 保持幅值守恒特性,无可见耗散
        \item 产生对称的寄生振荡,波形分裂为双峰结构
        \item 相位误差表现为波包中心位置的周期性偏移
    \end{itemize}
\end{itemize}

\subsection{综合对比}
\begin{table}[h]
    \centering
    \caption{数值格式特性总结}
    \begin{tabular}{l|ccc}
    特性         & L-W格式 & W-B格式 & Leap-frog \\ \hline
    CFL条件      & [0,1]   & [0,2]   & [0,1]     \\
    耗散性       & 弱      & 强      & 无        \\
    相位误差     & 滞后    & 超前    & 分裂      \\
    适用场景     & 光滑解  & 耗散控制 & 守恒系统  \\
    \end{tabular}
\end{table}

三种格式呈现显著不同的行为特征,与理论分析中的放大因子特性和差分格式构造方式一致。实际计算时应根据问题物理特性选择合适格式。
%附录
\newpage
\appendix
\section{AI工具使用声明表}
\begin{table}[H]
    \centering
    \begin{tabular}{c|c|c}
        \hline
        使用内容 & 工具名称 & 使用目的 \\ \hline
        hw3.tex 1-9行、图片插入 & Github Copilot & 调整pdf格式,调用宏包,省略插入图片的重复性工作 \\ 
        .gitignore & Github Copilot & 针对于python和latex的.gitignore文件,完全由Copilot生成  
    \end{tabular}
    \label{tab:AI_tools}
\end{table}
\end{document}